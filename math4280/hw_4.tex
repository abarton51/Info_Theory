\documentclass[10pt,twoside]{article}
\usepackage{amssymb, amsmath, amsthm, amsfonts, epsfig, graphicx, dsfont,
  bbm, bbold, url, color, setspace, multirow, pinlabel}
\usepackage[all]{xy}

\usepackage{fancyhdr} \setlength{\voffset}{-1in}
\setlength{\topmargin}{0in} \setlength{\textheight}{9.5in}
\setlength{\textwidth}{6.5in} \setlength{\hoffset}{0in}
\setlength{\oddsidemargin}{0in} \setlength{\evensidemargin}{0in}
\setlength{\marginparsep}{0in} \setlength{\marginparwidth}{0in}
\setlength{\headsep}{0.25in} \setlength{\headheight}{0.5in}
\pagestyle{fancy}

\newcommand{\copyrightnotice}{\enlargethispage{24pt}
\let\thefootnote\relax\footnote{\copyright 2022 Christopher Heil}}

\newcommand{\tildep}{{\hbox{\raise .08ex
\hbox{${\scriptstyle{\sim}}$}}}\hskip 0.4 pt}

\newcommand{\0}{\mathbf{0}}
\newcommand{\challenge}{{\hglue -3 pt *\ }}
\newcommand{\COLON}{\! : \!}
\newcommand{\EQ}{\; = \;}
\newcommand{\Eq}{\, = \,}
\newcommand{\NE}{\; \ne \;}
\newcommand{\Ne}{\, \ne \,}
\newcommand{\GE}{\; \ge \;}
\newcommand{\Ge}{\, \ge \,}
\newcommand{\GT}{\; > \;}
\newcommand{\Gt}{\, > \,}
\newcommand{\LE}{\; \le \;}
\newcommand{\Le}{\, \le \,}
\newcommand{\LT}{\; < \;}
\newcommand{\Lt}{\, < \,}
\newcommand{\TO}{\; \to \;}
\newcommand{\To}{\, \to \,}
\newcommand{\MAPSTO}{\; \mapsto \;}
\newcommand{\Capp}{\, \cap \,}
\newcommand{\CAP}{\; \cap \;}
\newcommand{\Cupp}{\, \cup \,}
\newcommand{\CUP}{\; \cup \;}
\newcommand{\MONOTONE}{\; \monotone \;}
\newcommand{\DECREASE}{\; \decrease \;}
\newcommand{\APPROX}{\; \approx \;}
\newcommand{\EQUIVNORM}{\; \asymp \;}
\newcommand{\Iff}{\; \iff \;}
\newcommand{\IFF}{\quad \iff \quad}
\newcommand{\Implies}{\; \implies \;}
\newcommand{\IMPLIES}{\quad \implies \quad}
\newcommand{\IN}{\; \in \;}
\newcommand{\In}{\, \in \,}
\newcommand{\NOTIN}{\; \not\in \;}
\newcommand{\NOTTO}{\; \notto \;}
\newcommand{\NOTIMPLIES}{\quad \Longrightarrow\hskip -14 pt / \hskip 4 pt \quad}
\newcommand{\oneway}{\quad
 \genfrac{}{}{0pt}{}{\implies}{\Longleftarrow\hskip -14 pt / \hskip 4 pt} \quad}
\newcommand{\SETMINUS}{\, \setminus \,}
\newcommand{\Setminus}{\tinyback \setminus \tinyback}
\newcommand{\SUBSET}{\; \subseteq \;}
\newcommand{\SUBSETEQ}{\; \subseteq \;}
\newcommand{\Subseteq}{\, \subseteq \,}
\newcommand{\SUBSETNEQ}{\; \subsetneq \;}
\newcommand{\Subsetneq}{\, \subsetneq \,}
\newcommand{\SUPSETEQ}{\; \supseteq \;}
\newcommand{\Supseteq}{\, \supseteq \,}
\newcommand{\SUPSETNEQ}{\; \supsetneq \;}
\newcommand{\Supsetneq}{\, \supsetneq \,}
\newcommand{\CONG}{\; \cong \;}
\newcommand{\plus}{\; + \;}
\newcommand{\Plus}{\, + \,}
\newcommand{\minus}{\; - \;}
\newcommand{\Minus}{\, - \,}
\newcommand{\completeimplies}{\underset {\text{if~$X$ is complete}} \IMPLIES}
\newcommand{\eqdef}{\overset {\raise .4ex \hbox{\text{\scriptsize def}}} =}

\newcommand{\redrightarrow}{{\color{red} \rightarrow}}
\newcommand{\rednearrow}{{\color{red} \nearrow}}
\newcommand{\rednwarrow}{{\color{red} \nwarrow}}
\newcommand{\redsearrow}{{\color{red} \searrow}}
\newcommand{\redswarrow}{{\color{red} \swarrow}}
\newcommand{\reddownarrow}{{\color{red} \downarrow}}
\newcommand{\redDownarrow}{\color{red} \Downarrow}
\newcommand{\redImplies}{{\color{red} \Implies} }

%\newcommand{\svee}{{\scriptscriptstyle{\vee}}}
\newcommand{\svee}{{\hbox{\raise .4ex \hbox{${\scriptscriptstyle{\vee}}$}}}}
\newcommand{\swedge}{{\hbox{\raise .4ex \hbox{${\scriptscriptstyle{\wedge}}$}}}}
\newcommand{\sveeexp}{\svee\tinyback}
\newcommand{\swedgeexp}{\swedge\tinyback}
\newcommand{\swedgehat}[1]{\overset {\scriptscriptstyle{\wedge}} #1}
\newcommand{\sveecheck}[1]{\overset {\scriptscriptstyle{\vee}} #1}

\newcommand{\Ac}{{\mathcal{A}}}
\newcommand{\AC}{\mathrm{AC}}
\newcommand{\avg}{\textup{Avg}}
\newcommand{\alphabar}{\overline{\alpha}}
\newcommand{\betabar}{\overline{\beta}}
\newcommand{\Bc}{{\mathcal{B}}}
%\newcommand{\ball}{{\mathrm{Ball}}}
%\newcommand{\Ball}{{\mathrm{Ball}}}
\newcommand{\ball}{D}
\newcommand{\littleback}{\hskip -0.6 true pt}
\newcommand{\tinyback}{\hskip -1 true pt}
\newcommand{\medback}{\hskip -1.2 true pt}
\newcommand{\bigback}{\hskip -1.5 true pt}
\newcommand{\bigabs}[1]{\bigl|#1\bigr|}
\newcommand{\Bigabs}[1]{\Bigl|#1\Bigr|}
\newcommand{\biggabs}[1]{\biggl|#1\biggr|}
\newcommand{\bigbracket}[1]{\bigl[#1\bigr]}
\newcommand{\Bigbracket}[1]{\Bigl[#1\Bigr]}
\newcommand{\biggbracket}[1]{\biggl[#1\biggr]}
\newcommand{\borel}{{\Bc_\sigma}}
\newcommand{\BV}{\mathrm{BV}}
\newcommand{\C}{\mathbb{C}}
\newcommand{\Cc}{{\mathcal{C}}}
\newcommand{\chat}{\widehat{c}\tinyspace}
\newcommand{\chatprime}{\chat^{\,\prime}}
%\newcommand{\ccheck}{\sveecheck{c}}
\newcommand{\ccheck}{\overset {\scriptscriptstyle{\hskip 1.5 pt \vee}} c}
\newcommand{\ccard}{{\mathfrak{c}}}
\newcommand{\card}{\textup{card}}
\newcommand{\ceiling}[1]{\lceil #1 \rceil}
\newcommand{\veecheck}[1]{\overset \vee #1}
\newcommand{\CHI}{\hbox{\raise .4ex \hbox{$\chi$}}}
\newcommand{\CHIA}{\CHI_{\hskip -0.7 true pt A}}
\newcommand{\CHIhat}{\widehat{\CHI}}
\newcommand{\CHINhat}
    {\CHI_{\!N} \hskip -10.5pt \widehat{\phantom{\CHI}} \hskip 3 pt}
\newcommand{\CHIcheck}
 {\overset {\lower .1ex \hbox{${\scriptscriptstyle{\hskip 1.5 pt\vee}}$}}{\CHI}}
\newcommand{\CHINcheck}
    {\overset {\scriptscriptstyle{\hskip 4.0 pt \vee}} \CHI_{\!N}}
\newcommand{\CHIOmegacheck}
    {\overset {\scriptscriptstyle{\hskip 2.5 pt \vee}} \CHI_{\!\varOmega}}
\newcommand{\closure}[1]{\overline{#1}}
\newcommand{\clspan}{{\overline{\mbox{\rm span}}}}
\newcommand{\codim}{{\textup{codim}}}
\newcommand{\Col}{\textup{Col}}
\newcommand{\comp}{{\mathrm{C}}}
\newcommand{\medcap}{\operatornamewithlimits{\textstyle\bigcap}}
\newcommand{\medcup}{\operatornamewithlimits{\textstyle\bigcup}}
\newcommand{\smallcup}{\operatornamewithlimits{\scriptstyle\bigcup}}
\newcommand{\scap}{\hbox{\raise .25ex
\hbox{${\operatornamewithlimits{\scriptstyle\bigcap}}$}}}
\newcommand{\scup}{\hbox{\raise .25ex
\hbox{${\operatornamewithlimits{\scriptstyle\bigcup}}$}}}
\newcommand{\ddx}{\frac{d}{dx}}
\newcommand{\Dc}{{\mathcal{D}}}
\newcommand{\decrease}{\searrow}
\newcommand{\deltahat}{\widehat{\delta}}
\newcommand{\deltacheck}
  {\overset {\lower .4ex \hbox{${\scriptscriptstyle{\hskip 2 pt\vee}}$}} \delta}
\newcommand{\di}{\displaystyle}
\newcommand{\diam}{{\textup{diam}}}
\newcommand{\dist}{{\textup{dist}}}
\newcommand{\divides}{{\tinyspace | \tinyspace}}
\newcommand{\notdivides}{{\tinyspace\nmid\tinyspace}}
\newcommand{\domain}{{\textup{domain}}}
\newcommand{\tE}{{\widetilde{E}}}
\newcommand{\Ec}{{\mathcal{E}}}
\newcommand{\Emptyset}{\varnothing}
\newcommand{\bigenumspace}{\vglue -4 pt}
\newcommand{\enumspace}{\vglue -3 pt}
\newcommand{\smallenumspace}{\vglue -2 pt}
\newcommand{\eps}{\varepsilon}
%\newcommand{\esssup}{\mathop{\text{\rm ess \, sup}}}
\newcommand{\esssup}{\operatornamewithlimits{{\textup{ess\tinyspace sup}}}}
\newcommand{\bigeval}[1]{{\bigl. #1 \bigr|}}
\newcommand{\Bigeval}[1]{{\Bigl. #1 \Bigr|}}
\newcommand{\biggeval}[1]{{\biggl. #1 \biggr|}}
%\newcommand{\F}{\mathbf{\overline{F}}}
\newcommand{\F}{\mathbf{F}}
%\newcommand{\F}{\mathbb{F}}
\newcommand{\Fc}{{\mathcal{F}}}
\newcommand{\tFc}{{\widetilde{\mathcal{F}}}}
\newcommand{\Fcheck}
    {\overset {\lower .4ex \hbox{${\scriptscriptstyle{\hskip 2 pt\vee}}$}} F}
\newcommand{\fcheck}{\fveecheck}
\newcommand{\fhat}{\widehat{f}}
%\newcommand{\fhat}{\overset {\scriptscriptstyle{\hskip 4 pt\wedge}} f}
\newcommand{\fwedgehat}
    {\overset {\lower .6ex \hbox{${\scriptscriptstyle{\hskip 3 pt\wedge}}$}} f}
\newcommand{\fveecheck}
    {\overset {\lower .4ex \hbox{${\scriptscriptstyle{\hskip 2 pt\vee}}$}} f}
\newcommand{\fkcheck}
   {\overset {\lower .4ex \hbox{${\scriptscriptstyle{\hskip 1 pt\vee}}$}} {f_k}}
%\newcommand{\fhatcheck}{\bigparen{\fwedgehat\tinyspace}^{\bigback\svee}}
\newcommand{\fhatcheck}{\bigparen{\fhat\biggerspace}^{\bigback\svee}}
\newcommand{\fcheckhat}{\bigparen{\fveecheck\tinyspace}^{\tinyback\swedge}}
\newcommand{\fhatprime}{\fhat^{\hskip 3pt\raiseprime}}
\newcommand{\raiseprime}{\hbox{\raise .3ex \hbox{${\scriptstyle{\prime}}$}}}
%\newcommand{\fhatprime}{\fhat^{\tinyspace\prime}}
\newcommand{\fprimehat}{\widehat{f'\littlespace}}
\newcommand{\flambdahat}
    {f_\lambda \hskip -8.5pt \widehat{\phantom{f}} \hskip 3 pt}
\newcommand{\tf}{\widetilde{f} \rule{0pt}{10.6pt}}
\newcommand{\tfh}{\widetilde{f}_{\hskip -0.2 true pt h}}
\newcommand{\tfhn}{\widetilde{f}_{\hskip -0.2 true pt h_n}}
%\newcommand{\tfh} {f_h \hskip -8pt \widetilde{\phantom{f}} \hskip 2 pt}
\newcommand{\fbar}{\bar{f} \rule{0pt}{10.4pt}}
%\newcommand{\tfhat}{\widehat{\widetilde{f}}}
\newcommand{\tfhat} {\widetilde{f}
    \hskip -4.5pt \widehat{\phantom{\widetilde{f}}} \hskip -1.5 pt}
\newcommand{\fkhat} {f_k \hskip -8pt \widehat{\phantom{f}} \hskip 2 pt}
\newcommand{\fmhat} {f_m \hskip -10pt \widehat{\phantom{f}} \hskip 4 pt}
\newcommand{\fnhat} {f_n \hskip -8.5pt \widehat{\phantom{f}} \hskip 2 pt}
\newcommand{\fNhat} {f_N \hskip -11.5pt \widehat{\phantom{M}} \hskip 4 pt}
\newcommand{\floor}[1]{\lfloor \tinyback #1 \tinyback \rfloor}
\newcommand{\Gc}{{\mathcal{G}}}
\newcommand{\Gcheck}
    {\overset {\lower .4ex \hbox{${\scriptscriptstyle{\hskip 2 pt\vee}}$}} G}
\newcommand{\ghat}{\widehat{g}}
%\newcommand{\gcheck}{\check{g}}
\newcommand{\ghatcheck}{\bigparen{\ghat\biggerspace}^{\bigback\svee}}
%\newcommand{\ghatcheck}{g^{\swedge\svee}}
\newcommand{\gcheck}{\gveecheck}
\newcommand{\gveecheck}
    {\overset {\lower .4ex \hbox{${\scriptscriptstyle{\hskip 2 pt\vee}}$}} g}
\newcommand{\gcheckhat}{\bigparen{\gveecheck\tinyspace}^{\tinyback\swedge}}
\newcommand{\gprimehat}{\tinyback\widehat{\,g'\,}\tinyback}
\newcommand{\glambdahat}
    {g_\lambda \hskip -8.5pt \widehat{\phantom{g}} \hskip 3 pt}
\newcommand{\glambdaprime}{g_\lambda^{\;\prime}}
\newcommand{\gnhat}{\widehat{g_n}}
\newcommand{\gncheck}
   {\overset {\lower .1ex \hbox{${\scriptscriptstyle{\hskip -3 pt\vee}}$}}{g_n}}
\newcommand{\tg}{\widetilde{g}}
%\newcommand{\tgh}{\widetilde{g}_{\hskip -0.1 true pt h}}
\newcommand{\tgh} {g_h \hskip -8pt \widetilde{\phantom{g}} \hskip 2 pt}
\newcommand{\Ghat}{\widehat{G}}
\newcommand{\graph}{{\textup{graph}}}
\newcommand{\wedgehat}[1]{\overset \wedge #1}
\newcommand{\tH}{\widetilde{H}}
\newcommand{\hhat}{\widehat{h}}
\newcommand{\Hhat}{\widehat{H}}
\newcommand{\Hc}{{\mathcal{H}}}
\newcommand{\Hf}{H \hskip -0.7 true pt f}
\newcommand{\hcheck}
    {\overset {\lower .4ex \hbox{${\scriptscriptstyle{\hskip 2 pt\vee}}$}} h}
\newcommand{\hmhat} {h_m \hskip -12.5pt \widehat{\phantom{h}} \hskip 4 pt}
\newcommand{\hnhat} {h_n \hskip -11pt \widehat{\phantom{h}} \hskip 4 pt}
\newcommand{\hncheck}
   {\overset {\lower .2ex \hbox{${\scriptscriptstyle{\hskip -4 pt\vee}}$}}{h_n}}
%\newcommand{\Th}{\widetilde{h}}
\newcommand{\Herm}{{\text{H}}}
\newcommand{\HS}{{\text{HS}}}
\newcommand{\Imag}{\textup{Im}}
\newcommand{\ip}[2]{\langle#1,#2\rangle}
\newcommand{\bigip}[2]{\bigl\langle #1, \, #2 \bigr\rangle}
\newcommand{\Bigip}[2]{\Bigl\langle #1, \, #2 \Bigr\rangle}
\newcommand{\biggip}[2]{\biggl\langle #1, \, #2 \biggr\rangle}
\newcommand{\boxip}[2]{[#1,#2]}
\newcommand{\itema}{{\item{\textup{(a)}}}}
\newcommand{\itemb}{\item{\textup{(b)}}}
\newcommand{\Khat}{\widehat{K}}
\newcommand{\Kcheck}
  {\overset {\lower .4ex \hbox{${\scriptscriptstyle{\hskip 2 pt\vee}}$}} K}
\newcommand{\khat}{\widehat{k}}
\newcommand{\kNhat}{\widehat{k_N}}
\newcommand{\tk}{\widetilde{k}}
\newcommand{\klambdahat}
    {k_\lambda \hskip -8.5pt \widehat{\phantom{k}} \hskip 3 pt}
%\newcommand{\klambdahat}{\overset {\tinyspace\wedge} {k_\lambda}}
\newcommand{\klambdatilde}
    {k_\lambda \hskip -8.5pt \widetilde{\phantom{k}} \hskip 3 pt
     \rule{0pt}{10.6pt}}
\newcommand{\kntilde} {
    k_n \hskip -8.5pt \widetilde{\phantom{k}} \hskip 3 pt}
\newcommand{\knhat} {k_n \hskip -10pt \widehat{\phantom{k}} \hskip 4 pt}
\newcommand{\kncheck}
   {\overset {\lower .2ex \hbox{${\scriptscriptstyle{\hskip 1 pt\vee}}$}} {k_n}}
\newcommand{\Lc}{{\mathcal{L}}}
\newcommand{\length}{{\textup{length}}}
\newcommand{\Lip}{\textup{Lip}}
\newcommand{\loc}{{\mathrm{loc}}}
\newcommand{\Lpt}{\widetilde{L^p}}
\newcommand{\Liminf}{\operatornamewithlimits{{\textup{lim\biggerspace inf}}}}
\newcommand{\Limsup}{\operatornamewithlimits{{\textup{lim\biggerspace sup}}}}
\newcommand{\textmax}{{\mathrm{max}}}
\newcommand{\textmin}{{\mathrm{min}}}
\newcommand{\Mbar}{\overline{M}}
\newcommand{\cM}{{\mathcal{M}}}
\newcommand{\Mf}{M \hskip -0.7 true pt f}
\newcommand{\bmat}[1]{\left[\begin{array}{rrrrrrr} #1 \end{array}\right]}
\newcommand{\bmatf}[1]{\left[\begin{array}{ccccccc} #1 \end{array}\right]}
\newcommand{\bmatcen}[1]{\left[\begin{array}{ccccccc} #1 \end{array}\right]}
\newcommand{\smallbmat}[1]{\left[\smallmatrix #1 \endsmallmatrix\right]}
\newcommand{\measure}{\operatorname{\overset m \to}} 
\newcommand{\metric}{\mathrm{d}}
\newcommand{\monotone}{\nearrow}
\newcommand{\tmonotone}{\tinyback\nearrow\tinyback}
\newcommand{\tmu}{\widetilde{\mu}}
\newcommand{\mucheck}
    {\overset {\lower .4ex \hbox{${\scriptscriptstyle{\hskip 2 pt\vee}}$}} \mu}
\newcommand{\muhat}{\widehat{\mu}}
\newcommand{\mubar}{\overline{\mu}}
\newcommand{\nuhat}{\widehat{\nu}}
\newcommand{\one}{\mathbf{1}}
\newcommand{\N}{\mathbb{N}}
\newcommand{\Nc}{\mathcal{N}}
\newcommand{\notimplies}{{\Longrightarrow \hskip -14 pt / \hskip 4 pt}}
\newcommand{\notto}{{\hskip 2 pt \rightarrow \hskip -10 pt / \hskip 6 pt}}
\newcommand{\norm}[1]{\|#1\|}
\newcommand{\bignorm}[1]{\bigl\|#1\bigr\|}
\newcommand{\Bignorm}[1]{\Bigl\|#1\Bigr\|}
\newcommand{\biggnorm}[1]{\biggl\|#1\biggr\|}
\newcommand{\tnu}{\widetilde{\nu}}
\newcommand{\Oc}{{\mathcal{O}}}
\newcommand{\medoplus}{\operatornamewithlimits{\textstyle\bigoplus}}
\newcommand{\Omegab}{\mathbf{\varOmega}}
\newcommand{\osc}{{\textup{osc}}}
\newcommand{\Pb}{\mathbb{P}}
\newcommand{\Pbb}{{\mathbf{P}}}
\newcommand{\Pc}{\mathcal{P}}
\newcommand{\pv}{{\textup{pv}}}
\newcommand{\bigparen}[1]{\bigl(#1\bigr)}
\newcommand{\Bigparen}[1]{\Bigl(#1\Bigr)}
\newcommand{\biggparen}[1]{\biggl(#1\biggr)}
\newcommand{\Biggparen}[1]{\Biggl(#1\Biggr)}
\newcommand{\parensize}[1]{\left(#1\right)}
\newcommand{\phat}{\widehat{p}\tinyspace}
\newcommand{\phihat}{\widehat{\phi}}
\newcommand{\poly}{{\mathrm{poly}}}
\newcommand{\varphihat}{\widehat{\varphi}}
\newcommand{\tvarphi}{\widetilde{\varphi} \rule{0pt}{9.8pt} }
\newcommand{\varphixihat}
    {\varphi_{\xi} \hskip -9pt \widehat{\phantom{\varphi}} \hskip 2.5 pt}
\newcommand{\varphixikhat}
    {\varphi_{\xi_k} \hskip -9pt \widehat{\phantom{\varphi}} \hskip 2.5 pt}
\newcommand{\varphixitilde}
    {\varphi_{\xi} \hskip -9pt \widetilde{\phantom{\varphi}} \hskip 2.5 pt}
\newcommand{\varphionehat}
    {\varphi_{1} \hskip -9.5pt \widehat{\phantom{\varphi}} \hskip 1.5 pt}
\newcommand{\varphimhat}
    {\varphi_m \hskip -9pt \widehat{\phantom{\varphi}} \hskip 2.5 pt}
\newcommand{\varphinhat}
    {\varphi_n \hskip -10.0pt \widehat{\phantom{\varphi}} \hskip 2.5 pt}
\newcommand{\varphicheck}
    {\overset {\lower .4ex \hbox{${\scriptscriptstyle{\hskip 1 pt\vee}}$}}
              \varphi}
\newcommand{\psihat}{\widehat{\psi}}
\newcommand{\psixihat}
    {\psi_{\xi} \hskip -9pt \widehat{\phantom{\psi}} \hskip 2.5 pt}
\newcommand{\Q}{\mathbb{Q}}
\newcommand{\Qc}{{\mathcal{Q}}}
\newcommand{\tQ}{{\widetilde{Q}}}
\newcommand{\qeddef}{{\quad $\diamondsuit$}}
%\newcommand{\qeddef}{{$\ \hfill \diamondsuit$}}
\newcommand{\qeddeff}{{\qquad \diamondsuit}}
\newcommand{\qeddefff}{{\quad \diamondsuit}}
\newcommand{\R}{\mathbb{R}}
\newcommand{\Rc}{{\mathcal{R}}}
\newcommand{\Rhat}{\widehat{\R}}
%\newcommand{\Rbar}{{\bar{R}}}
\newcommand{\Rbar}
  {{\overset {\hskip -0.9 pt \lower 1.5pt \hbox{{\rule{6.7pt}{0.45pt}}}} \R}}
\newcommand{\subRbar}
   {{\overset {\hskip -0.8 pt \lower 1.5pt \hbox{{\rule{4.5pt}{0.5pt}}}} \R}}
\newcommand{\Real}{\textup{\text{Re}}}
\newcommand{\radius}{{\textup{radius}}}
\newcommand{\range}{{\textup{range}}}
\newcommand{\clrange}{{\overline{\mbox{\rm range}}}}
\newcommand{\rank}{{\textup{rank}}}
\newcommand{\row}{{\text{---}}}
%\newcommand{\shah}{\operatorname{\text{\cyr Sh}}}
\newcommand{\scheck}
    {\overset {\lower .4ex \hbox{${\scriptscriptstyle{\hskip 0.5 pt\vee}}$}} s}
\newcommand{\Sc}{{\mathcal{S}}}
\newcommand{\bS}{\bar{S}}
\newcommand{\tS}{\widetilde{S}}
%\newcommand{\SNa}{S_N^{\hskip 0.5 pt \hbox{\small a}}}
\newcommand{\SNa}{S_N^{\hskip 0.5 pt
  \hbox{\raise .3ex \hbox{\small\textup{a}}}}}
\newcommand{\SNaa}[1]{S_#1^{\hskip 0.5 pt
  \hbox{\raise .3ex \hbox{\small\textup{a}}}}}
%\newcommand{\SNa}{S_N^{\hskip 0.5 pt a}}
\newcommand{\SNo}{S_N^{\hskip 0.5 pt
  \hbox{\raise .3ex \hbox{\small\textup{o}}}}}
%\newcommand{\SNo}{S_N^{\hskip 0.5 pt \hbox{o}}}
%\newcommand{\SNo}{S_N^{\hskip 1 pt o}}
\newcommand{\SNt}{S_N^{\hskip 0.5 pt
  \hbox{\raise .3ex \hbox{\small\textup{t}}}}}
\newcommand{\SN}{S_N \hskip -0.6 pt}
\newcommand{\set}[1]{\{#1\}}
\newcommand{\bigset}[1]{\bigl\{#1\bigr\}}
\newcommand{\Bigset}[1]{\Bigl\{#1\Bigr\}}
\newcommand{\biggset}[1]{\biggl\{#1\biggr\}}
\newcommand{\setsize}[1]{\left\{#1\right\}}
\newcommand{\Sigmabar}{\overline{\varSigma}}
\newcommand{\sgn}{{\mathrm{sgn}}}
\newcommand{\sign}{{\mathrm{sign}}}
\newcommand{\sinc}{{\mathrm{sinc}}}
\newcommand{\sskip}[1]{{\vskip 0.#1 true in}}
\newcommand{\smallestskip}{{\vskip 1 pt}}
\newcommand{\smallerskip}{{\vskip 2 pt}}
\newcommand{\itsyspace}{\hskip 0.4 pt}
\newcommand{\tiniestspace}{\hskip 0.6 pt}
\newcommand{\tinierspace}{\hskip 0.8 pt}
\newcommand{\tinyspace}{\hskip 1 pt}
\newcommand{\littlespace}{\hskip 1.2 pt}
\newcommand{\bigspace}{\hskip 1.5 pt}
\newcommand{\biggerspace}{\hskip 2 pt}
%\newcommand{\solution}{\vskip 0.1 true in \noindent
%\underbar{Solution} \newline \indent}
%\newcommand{\solutionf}{\vskip 0.1 true in \noindent \underbar{Solution}}
%\newcommand{\solutionp}{\noindent \underbar{Solution} \newline \indent}
%\newcommand{\booksolution}{\noindent{\it Solution}. }
\newcommand{\Span}{\mathrm{span}}
\newcommand{\spectrum}{\mathrm{Sp}}
%\newcommand{\subset}{\subseteq}
\newcommand{\supp}{{\textup{supp}}}
\newcommand{\trans}{{\text{T}}}
\newcommand{\T}{{\mathbb{T}}}
\newcommand{\Tb}{{\mathbf{T}}}
\newcommand{\Tc}{{\mathcal{T}}}
\newcommand{\Tt}{{\textup{T}}}
\newcommand{\Top}{{\mathcal{T}}}
\newcommand{\takeaway}{\hskip 0.8 pt \backslash \hskip 0.8 pt}
\newcommand{\tensor}{\otimes}
\newcommand{\textm}{{\mathrm{m}}}
\newcommand{\thetahat}{\widehat{\theta}}
\newcommand{\thetalambdahat}{\widehat{\theta_\lambda}}
\newcommand{\thetaMhat}{\widehat{\theta_M}}
\newcommand{\transpose}{{\mathrm{T}}}
\newcommand{\trace}{{\mathrm{trace}}}
\newcommand{\spaceback}{\hskip -1.2 true pt}
\newcommand{\tripnorm}[1]{{{|}\spaceback{|}\spaceback{|} #1
                           {|}\spaceback{|}\spaceback{|}}}
%\newcommand{\tripnorm}[1]{{{|}\!{|}\!{|} #1 {|}\!{|}\!{|}}}
\newcommand{\quadnorm}[1]{{{|}\spaceback{|}\spaceback{|}\spaceback{|} #1
                           {|}\spaceback{|}\spaceback{|}\spaceback{|}}}
\newcommand{\Uc}{{\mathcal{U}}}
\newcommand{\uni}{{\mathrm{u}}}
\newcommand{\vhat}{\widehat{v}}
\newcommand{\vlambdahat}
    {v_\lambda \hskip -8.5pt \widehat{\phantom{v}} \hskip 2.5 pt}
\newcommand{\itsyvertrule}{\rule{0pt}{5.2pt}}
\newcommand{\tiniestvertrule}{\rule{0pt}{6.8pt}}
\newcommand{\tiniervertrule}{\rule{0pt}{8.2pt}}
\newcommand{\tinyvertrule}{\rule{0pt}{8.8pt}}
\newcommand{\smallvertrule}{\rule{0pt}{9.5pt}}
\newcommand{\smallervertrule}{\rule{0pt}{9.6pt}}
\newcommand{\medvertrule}{\rule{0pt}{9.8pt}}
\newcommand{\meddervertrule}{\rule{0pt}{10.1pt}}
\newcommand{\vertrule}{\rule{0pt}{10.4pt}}
\newcommand{\bigvertrule}{\rule{0pt}{10.8pt}}
\newcommand{\biggervertrule}{\rule{0pt}{11.8pt}}
\newcommand{\biggestvertrule}{\rule{0pt}{12.8pt}}
\newcommand{\Bigvertrule}{\rule{0pt}{13.2pt}}
%\newcommand{\vertspace}{\phantom{\widehat{\Sigma}\!}}
\newcommand{\vertspace}{\phantom{\Bigparen{X}\!}}
\newcommand{\bigvertspace}{\phantom{\Biggparen{X}}\!\!}
\newcommand{\vol}{{\mathrm{vol}}}
\newcommand{\what}{\widehat{w}}
%\newcommand{\wcheck}{\sveecheck{w}}
\newcommand{\wcheck}{\overset {\scriptscriptstyle{\hskip 1.5 pt \vee}} w}
\newcommand{\Wc}{{\mathcal{W}}}
\newcommand{\Wchat}{\widehat{\Wc}}
%\newcommand{\What}{\swedgehat{W}}
\newcommand{\What}{W \hskip -9.0pt \widehat{\phantom{T}} \hskip 1.5 pt}
\newcommand{\Wcheck}{\sveecheck{W}}
\newcommand{\wlambdahat}
    {w_\lambda \hskip -10pt \widehat{\phantom{w}} \hskip 2.5 pt}
\newcommand{\WsubN}{W_{\!N}\tinyback}
\newcommand{\WN}{W_{\tinyback N}}
%\newcommand{\weak}{\operatorname{\overset w \to \to}} 
\newcommand{\weak}{\operatorname{\overset w \to}} 
\newcommand{\notweak}{{\weak \hskip -9 pt / \hskip 8 pt}}
\newcommand{\weakstar}{\operatorname{\overset {w\text{*}} \longrightarrow}} 
%\newcommand{\weakstar}{\operatorname{\overset w\text{*} \to \to}} 
\newcommand{\Weak}{\textup{Weak-}}
\newcommand{\widecheck}[1]{\overset{\vee}{#1}}
\newcommand{\bx}{\mathbf{x}}
\newcommand{\xhat}{\widehat{x}}
\newcommand{\xnhat} {x_n \hskip -9pt \widehat{\phantom{x}} \hskip 3 pt}
\newcommand{\tx}{\widetilde{x}}
\newcommand{\tX}{\widetilde{X}}
\newcommand{\X}{X\!}
\newcommand{\Y}{Y\!}
\newcommand{\Z}{\mathbb{Z}}
\newcommand{\Zc}{{\mathcal{Z}}}
\newcommand{\zeroveecheck}
    {\overset {\lower .4ex \hbox{${\scriptscriptstyle{\hskip 0.5 pt\vee}}$}} 0}

%\newcommand{\indexspacer}{\rule{0pt}{10.6pt}}
%\newcommand{\indexspacer}{}

\newcommand{\indexspaced}{\rule{0pt}{11.8pt}}
%\newcommand{\bigindexspacer}{\rule{0pt}{13.9pt}}
\newcommand{\bigindexspacer}{\rule{0pt}{13.1pt}}

\hyphenation{Bun-ya-kow-ski}
\hyphenation{spatial}

\newcommand{\solution}{\vskip 0.1 true in \noindent
\underbar{Solution} \newline \indent}
\newcommand{\solutionf}{\vskip 0.1 true in \noindent \underbar{Solution}}
\newcommand{\solutionp}{\noindent \underbar{Solution} \newline \indent}
\newcommand{\booksolution}{\noindent{\it Solution}. }


\newcommand{\Pro}{\ensuremath{\mathbb{P}}}
\newcommand{\condPro}[2]{\ensuremath{\mathbb{P}}(#1 $|$ #2)}
\newcommand{\E}{\ensuremath{\mathbb{E}}}

\onehalfspace

\fancyhead[LO,LE]{{MATH 4280 - Information Theory - Professor Wang} \fancyhead[RO,RE]{Due 03/24/2023 at 11:59 pm}}
\chead{\textbf{}} \cfoot{}
\fancyfoot[LO,LE]{} \fancyfoot[RO,RE]{Page \thepage\ of
  \pageref{LastPage}} \renewcommand{\footrulewidth}{0.5pt}
\parindent 0in
% ------------------------------------------------------%
% -------------------Begin Document---------------------%
% ------------------------------------------------------%
\begin{document}

\begin{center}
\huge{\bf{Homework 4} - Austin Barton}
\end{center}

\medskip

\noindent \large{\textbf{Collaborators: N/A}}

\medskip

\begin{itemize}
    \item\textbf{Problem 1:} \newline
    \noindent\makebox[\linewidth]{\rule{18cm}{0.4pt}}
    An $n\times n$ matrix $P = [P_{ij}]$ is said to be doubly stochastic if $P_{ij}\geq 0$ and $\sum_{j}P_{ij} = 1$ for all $i\in [n]$ and $\sum_{i}P_{ij} = 1$ for all $j\in [n]$. An $n\times n$ matrix $P$ is said to be a permutation matrix if it is doubly stochastic and there is precisely one $P_{ij} = 1$ in each row and each column. It can be shown that every doubly stochastic matrix can be written as the convex combination of permutation matrices.
    \begin{enumerate}
        \item Let $\mathbf{a}^t = (a_1, a_2, \ldots, a_n)$ be a probability vector, and let $\mathbf{b} = \mathbf{a}P$, where $P$ is doubly stochastic. Show that $\mathbf{b}$ is a probability vector and that 
        \begin{gather*}
            H(b_1, b_2, \ldots, b_n)\geq H(a_1, a_2, \ldots, a_n)
        \end{gather*}
        \item Show that a stationary distribution $\mu$ for a doubly stochastic matrix $P$ is the uniform distribution.
        \item Conversely, prove that if the uniform distribution is a stationary distribution for a Markov transition matrix $P$, then $P$ is doubly stochastic.
    \end{enumerate}
    \begin{proof}[Answers.]
    \begin{enumerate}
        \item \begin{proof}[Answer.]
            First, to show $\mathbf{b}$ is a probability vector. Consider,
            \begin{gather*}
                (\mathbf{a}P)_j = b_j = \sum_{i=1}^n a_iP_{ij}
            \end{gather*}
            For $j=1,\ldots, n$. But \begin{gather*}
                \sum_{j=1}^n b_j = \sum_{j=1}^n   \sum_{i=1}^n a_iP_{ij} = \sum_{i=1}^n\sum_{j=1}^n a_iP_{ij} \\
                = \sum_{i=1}^na_i\sum_{j=1}^nP_{ij} = \sum_{i=1}^na_i = 1
            \end{gather*}
            Since $P$ is doubly stochastic and $\mathbf{a}$ is a probability vector. Thus, $\mathbf{b}$ is a probability vector. Now, to show that $H(\mathbf{b})\geq H(\mathbf{a})$. Consider, 
            \begin{gather*}
                H(\mathbf{b}) - H(\mathbf{a}) = H(b_1,\ldots, b_n) - H(a_1, \ldots, a_n) \\
                = -\sum_{j=1}^n b_j\log b_j + \sum_{i=1}^n a_i\log a_i
            \end{gather*}
            We know each $b_j = \sum_{i=1}^n a_iP_{ij}$. So,
            \begin{gather*}
                -\sum_{j=1}^n b_j\log b_j + \sum_{i=1}^n a_i\log a_i = -\sum_{j=1}^n\sum_{i=1}^n a_iP_{ij}\log \sum_{i'=1}^n a_{i'}P_{i'j} + \sum_{i=1}^n a_i \log a_i \\
                = \sum_{i=1}^n \sum_{j=1}^n a_iP_{ij}\log \frac{a_i}{\sum_{i'=1}^n a_{i'}P_{i'j}} = \sum_{i=1}^n \sum_{j=1}^n a_iP_{ij}\log \frac{a_i}{b_j}
            \end{gather*}
            Now, using the log-sum inequality from Chapter 2,
            \begin{gather*}
                \sum_{i=1}^n \sum_{j=1}^n a_iP_{ij}\log \frac{a_i}{b_j} \geq \Big(\sum_{j=1}^n\sum_{i=1}^n a_iP_{ij} \Big) \log \frac{\sum_{j=1}^n\sum_{i=1}^n a_i}{\sum_{j=1}^n\sum_{i=1}^n b_j} \\
                = \Big(\sum_{j=1}^n\sum_{i=1}^n a_iP_{ij} \Big) \log \frac{\sum_{j=1}^n\sum_{i=1}^n a_i}{\sum_{i=1}^n\sum_{j=1}^nb_j} \\
                = \Big(\sum_{j=1}^n\sum_{i=1}^n a_iP_{ij} \Big) \log \frac{\sum_{j=1}^n 1}{\sum_{i=1}^n 1} =  \Big(\sum_{j=1}^n\sum_{i=1}^n a_iP_{ij} \Big)\log \frac{n}{n} = \Big(\sum_{j=1}^n\sum_{i=1}^n a_iP_{ij} \Big)\log 1 = 0
            \end{gather*}
            Thus, 
            \begin{gather*}
                H(b_1, \ldots, b_n) \geq H(a_1, \ldots, a_n)
            \end{gather*}
        \end{proof}
        \item \begin{proof}[Answer.]
            We want to show that $\mu = (\mu_1, \ldots, \mu_n) = (\frac{1}{n}, \ldots, \frac{1}{n})$
            \iffalse
            \begin{gather*}
                \mu = 
                \begin{bmatrix}
                    \mu_1 \\
                    \vdots \\
                    \mu_n
                \end{bmatrix}
                =
                \begin{bmatrix}
                \frac{1}{n} \\
                \vdots \\
                \frac{1}{n}
                \end{bmatrix}
            \end{gather*}
            \fi
            is a stationary distribution for a doubly stochastic matrix $P$. Let $P = [P_1 \ldots P_n]$ where each $P_j$ for $j=1,\ldots, n$ is the $j^{th}$ column vector of $P$. Consider, 
            \begin{gather*}
                \mu P = \mu[P_1 \ldots P_n] = [\mu P_1\ldots, \mu P_n] 
            \end{gather*}
            And so for each $i=1,\ldots, n$,
            \begin{gather*}
                (\mu P)_i = \sum_{j=1}^n\mu_i P_j =  \mu_i\sum_{j=1}^n P_j = \mu_i(1) = \mu_i
            \end{gather*}
            Thus, the uniform distribution is a stationary distribution for a doubly stochastic matrix $P$.
        \end{proof}
        \item \begin{proof}[Answer.]
            Let $\mu$ be a uniform distribution that is a stationary distribution for a Markov transition matrix $P$. We want to show that $P$ is doubly stochastic.\smallskip

            Since $P$ is a Markov transition matrix, we know that each $P_{ij}\geq 0$ and $\sum_{j=1}^n P_{ij} = 1$.

            Consider, 
            \begin{gather*}
                \mu P = 
                \mu
                [P_1 \ldots P_n] = [\mu P_1 \ldots \mu P_n]
            \end{gather*}
            
            Each $j^{th}$ column of $\mu P$ is
            \begin{gather*}
                (\mu P)_j = \sum_{i=1}^n \mu_j P_{i j} = \mu_j
            \end{gather*}
            since $\mu$ is a stationary distribution. Since $\mu$ is a uniform distribution, this is,
            \begin{gather*}
                \mu_j = \frac{1}{n}
            \end{gather*}
            Thus, 
            \begin{gather*}
                \frac{1}{n}\sum_{i=1}^n P_{ij} = \frac{1}{n} \\
                \Rightarrow \sum_{i=1}^n P_{ij} = 1
            \end{gather*}
            Therefore, for each $j=1,\ldots, n$, we have that,
            \begin{gather*}
                \sum_{i=1}^n P_{ij} = 1
            \end{gather*}
           Therefore, $P$ is a doubly stochastic matrix by definition.
            
            
        \end{proof}

        
    \end{enumerate}
    \end{proof}
    
    \item\textbf{Problem 2:} \newline
    \noindent\makebox[\linewidth]{\rule{18cm}{0.4pt}}
    \begin{proof}[Answer to part a).]
    Consider, 
    \begin{gather*}
        \frac{H(X_1, \ldots, X_n)}{n} = \frac{\sum_{i=1}^n H(X_i|X_{i-1}, \ldots, X_1)}{n} \\
        = \frac{H(X_n|X_{n-1}, \ldots, X_1) + \sum_{i=1}^{n-1} H(X_{i}|X_{i-1}, \ldots, X_1)}{n}
    \end{gather*}
    by the chain rule for entropy. By the chain rule again we have,
    \begin{gather*}
        = \frac{H(X_n|X_{n-1},\ldots, X_1) + H(X_1, \ldots, X_{n-1})}{n}
    \end{gather*}
    Since this is a stationary stochastic process, we know that $H(X_n|X_{n-1}, \ldots, X_1)$ is non-increasing in $n$. That is, for each $i = 1, \ldots, n$,
    \begin{gather*}
        H(X_n|X_{n-1}, \ldots, X_1) \leq H(X_{i}|X_{i-1
        }, \ldots, X_1)
    \end{gather*}
    So then,
    \begin{gather*}
        H(X_n|X_{n-1}, \ldots, X_1) \leq \frac{\sum_{i=1}^{n-1}H(X_{i}|X_{i-1}, \ldots, X_1)}{n-1}
    \end{gather*}
    since the right hand side is averaged over terms that are all greater than or equal to the left hand side. Thus, by the chain rule for entropy in the numerator of the right hand side,
    \begin{gather*}
        H(X_n|X_{n-1}, \ldots, X_1)\leq \frac{H(X_1, \ldots, X_{n-1})}{n-1}
    \end{gather*}
    So, we have that 
    \begin{gather*}
         \frac{H(X_1, \ldots, X_n)}{n} = \frac{H(X_n|X_{n-1},\ldots, X_1) + H(X_1, \ldots, X_{n-1})}{n} \\
         \text{and} \\
         H(X_n|X_{n-1}, \ldots, X_1) \leq \frac{H(X_1, \ldots, X_{n-1})}{n-1}
    \end{gather*}
    Thus, 
    \begin{gather*}
        \frac{H(X_1, \ldots, X_n)}{n} \leq \frac{\frac{1}{n-1}H(X_1, \ldots, X_{n-1}) + H(X_1, \ldots, X_{n-1})}{n} \\
        = \frac{H(X_1, \ldots, X_{n-1}) + (n-1)H(X_1, \ldots, X_{n-1})}{n(n-1)} = \frac{nH(X_1, \ldots, X_{n-1})}{n(n-1)} \\
        = \frac{H(X_1, \ldots, X_{n-1})}{n-1}
    \end{gather*}
    That is, 
    \begin{gather*}
        \frac{H(X_1, \ldots, X_n)}{n} \leq \frac{H(X_1, \ldots, X_{n-1})}{n-1}
    \end{gather*}
    Thus, it has been shown.
    \end{proof}
    \begin{proof}[Answer to part b)]
        For each $i=1, \ldots, n$, since this is a stationary stochastic process, we have that,
        \begin{gather*}
            H(X_n|X_{n-1}, \ldots, X_1) \leq H(X_i|X_{i-1}, \ldots, X_1)
        \end{gather*}
        Then the mean 
        \begin{gather*}
            \frac{\sum_{i=1}H(X_i|X_{i-1}, \ldots, X_1)}{n}
        \end{gather*}
        is bounded below by $H(X_n|X_{n-1}, \ldots, X_1)$ since for each $i=1,\ldots, n$, we have that $H(X_i|X_{i-1}, \ldots, X_1)\geq H(X_n|X_{n-1}, \ldots, X_1)\geq 0$. Thus, 
        \begin{gather*}
              H(X_n|X_{n-1}, \ldots, X_1)\leq \frac{\sum_{i=1}H(X_i|X_{i-1}, \ldots, X_1)}{n}
        \end{gather*}By the chain rule for entropy, this is,
        \begin{gather*}
            H(X_n|X_{n-1}, \ldots, X_1) \leq \frac{H(X_1, \ldots, X_n)}{n}
        \end{gather*}
        as required.
    \end{proof}
    
    \item\textbf{Problem 3:} \newline
    \noindent\makebox[\linewidth]{\rule{18cm}{0.4pt}}
    
    \begin{proof}[Answer.]
    Following the same procedure from page $73$ to find the stationary distribution, let $\mu$ be the stationary distribution vector. Then the stationary probability can be found by solving $\mu = \mu P$. Applying this to our transition matrix,
    \begin{gather*}
        \begin{bmatrix}
            \mu_1 & \mu_2
        \end{bmatrix}
        \begin{bmatrix}
            1 - p  & p \\
            q & 1-q
        \end{bmatrix} = 
        \begin{bmatrix}
            \mu_1 - p\mu_1 + q\mu_2 \\
            p\mu_1 + \mu_2 - q\mu_2
        \end{bmatrix} \\
        \Rightarrow
        \mu_1 = \frac{q}{p+q},\hspace{8pt}
        \mu_2 = \frac{p}{p+q}
    \end{gather*}
    Let $X_1\sim \mu$. The entropy rate is then calculated as,
    \begin{gather*}
        H(\chi) = H(X_2|X_1) \\
        = - \sum_{i, j}\mu_i P_{ij}\log P_{ij} \\
        = \frac{q}{p+q}H(p) + \frac{p}{p+q}H(q) \\
    \end{gather*}
    Now, we want to maximize $H(\chi) = H(X_2|X_1)$. Recall that the entropy of a random variable is maximized for the uniform distribution. Thus, $H(X_2|X_1)$ is maximized when $X_2|X_1$ is the uniform distribution. Additionally, since this stochastic process only has two states, the entropy rate is at most 1 bit. Intuitively, it requires only one binary question in order to describe the state of the system. The uniform distribution $p=1/2$ and $q=1/2$ satisfies this entropy rate.
    
    \end{proof}

\newpage
    
    \item\textbf{Problem 4:} \newline
    \noindent\makebox[\linewidth]{\rule{18cm}{0.4pt}}
    \begin{proof}[Answer.]
    Consider, 
    \begin{gather*}
        I(X_1, X_2, \ldots, X_n; X_{n+1}, X_{n+2}, \ldots, X_{2n}) = H(X_1, \ldots, X_n) - H(X_1, \ldots, X_n|X_{n+1}, \ldots, X_{2n}) \\
        = H(X_1, \ldots, X_n) + H(X_{n+1}, \ldots, X_{2n}) - H(X_1, \ldots, X_n, X_{n+1}, \ldots,  X_{2n})
    \end{gather*}
    By Theorem 2.4.1. Since this is a stationary stochastic process, it is invariant to shifts in the time index. That is, $H(X_1, \ldots, X_n) = H(X_{n+1}, \ldots, X_{2n})$. Thus, 
    \begin{gather*}
        I(X_1, X_2, \ldots, X_n; X_{n+1}, X_{n+2}, \ldots, X_{2n}) = 2H(X_1, \ldots, X_n) - H(X_1, \ldots, X_n, X_{n+1}, \ldots,  X_{2n})
    \end{gather*}
    Consider, $H(\chi)$. We know this exists since the process is stationary. That is, 
    \begin{gather*}
        H(\chi) = \lim_{n\to \infty} \frac{1}{n}H(X_1, \ldots, X_n)
    \end{gather*}
    exists. Thus, 
    \begin{gather*}
        \lim_{n\to\infty} \frac{1}{2n}2H(X_1, \ldots, X_n) = \lim_{n\to \infty} \frac{1}{n}H(X_1, \ldots, X_n) =  H(\chi)
    \end{gather*}
    Now consider, $\lim_{n\to \infty}\frac{1}{2n}H(X_1, \ldots, X_{2n})$. By the uniqueness of limits, we know that 
    \begin{gather*}
        \lim_{n\to\infty}\frac{1}{n}H(X_1, \ldots, X_n) = \lim_{n\to\infty} \frac{1}{2n}H(X_1, \ldots, X_{2n})
    \end{gather*}
    Thus, \begin{gather*}
        \lim_{n\to\infty} \frac{1}{2n}H(X_1, \ldots, X_{2n}) = H(\chi)
    \end{gather*}
    Therefore, 
    \begin{gather*}
        \lim_{n\to\infty}\frac{1}{2n} I(X_1, X_2, \ldots, X_n; X_{n+1}, X_{n+2}, \ldots, X_{2n}) \\
        = \lim_{n\to\infty} \frac{1}{2n}2H(X_1, \ldots, X_n) - \lim_{n\to\infty}\frac{1}{2n}H(X_1, \ldots, X_{2n}) \\
        = H(\chi) - H(\chi) = 0
    \end{gather*}
    as required.
    
    \end{proof}
    
    \item\textbf{Problem 5:} \newline
    \noindent\makebox[\linewidth]{\rule{18cm}{0.4pt}}
    \begin{proof}[Answer.]
    
    \end{proof}
\end{itemize}

\label{LastPage}
\end{document}